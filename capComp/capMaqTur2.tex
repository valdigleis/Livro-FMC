% Seta a imagem do capítulo
\chapterimage{chapter_head_2.pdf}
% O título e rótulo do cabiluto
\chapter{Turing Computabilidade: Parte 1}\label{cap:LLivresContexto}


\epigraph{``Eu acredito que às vezes são as pessoas que ninguém espera nada que fazem as coisas que ninguém consegue imaginar. ''}{Alan M. Turing.}

 
\section{Noções Fundamentais}
 
Neste primeiro momento para o estudo dos autômatos finitos será apresentando alguns conceitos fundamentais de extrema importância para o desenvolvimento das próximas seções e capítulos.

\begin{definition}[Alfabetos e Palavras]
	\cite{valdi2016master} Qualquer conjunto finito e não vazio $\Sigma$ será chamado de alfabeto. Qualquer sequência finita de símbolos na forma $a_1a_2\cdots a_n$ com $a_i \in \Sigma$ para todo $1 \leq i \leq n$ será chamada de palavra sobre o alfabeto $\Sigma$.
\end{definition}

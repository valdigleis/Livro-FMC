% Seta a imagem do capítulo
\chapterimage{chapter_head_2.pdf}
% O título e rótulo do cabiluto
\chapter{Computabilidade: Autômatos Finitos}\label{cap:LRegulares}


\epigraph{``Eu acredito que às vezes são as pessoas que ninguém espera nada que fazem as coisas que ninguém consegue imaginar. ''}{Alan M. Turing.}
 
\section{Noções Fundamentais}
 
Neste primeiro momento para o estudo dos autômatos finitos serão apresentados alguns conceitos fundamentais de extrema importância para o desenvolvimento das próximas seções e capítulos.

\begin{definition}[Alfabetos e Palavras]\label{def:AlfabetoPalavra}
	\cite{valdi2016master} Qualquer conjunto finito e não vazio $\Sigma$ será chamado de alfabeto. Qualquer sequência finita de símbolos na forma $a_1\cdots a_n$ com $a_i \in \Sigma$ para todo $1 \leq i \leq n$ será chamada de palavra sobre o alfabeto $\Sigma$.
\end{definition}


\begin{exem}
	Os conjuntos $\{0, 1, 2, 3\}, \{a, b, c\}, \{\heartsuit,\spadesuit, \Diamond, \clubsuit\}$ e $\{n \in \mathbb{N} \mid n \leq 25\}$ são todos alfabetos, os conjuntos $\mathbb{N}$ e $\mathbb{R}$ não são alfabetos.
\end{exem}

\begin{exem}
	Dado o alfabeto $\Sigma = \{0, 1, 2, 3\}$ tem-se que as sequências 0123, 102345, 1 e 0000 são todas palavras sobre $\Sigma$.
\end{exem}

\begin{definition}[Comprimento das palavras]\label{def:ComprimentoPalavra}
	Seja $w$ uma palavra qualquer sobre um certo alfabeto $\Sigma$, o comprimento\footnote{Dado a notação em alguns texto é usado o termo módulo em vez de usar comprimento.} de $w$, denotado por $|w|$, corresponde ao número de símbolos existentes em $w$.
\end{definition}

\begin{exem}
	Dado o alfabeto $\Sigma = \{a, b, c, d\}$ e as palavras $abcd, aacbd, c$ e $ddaacc$ tem-se que: $|abcd| = 4, |aa| = 2, |c| = 1$ e $|ddaacc| = 6$
\end{exem}

\begin{rema}
	Em especial quando $|w| = 1$, é dito que $w$ é uma palavra unitária, isto é, a mesma contém apenas um único símbolo do alfabeto.
\end{rema}

Como muito bem explicado em \cite{benjaLivro2010, hopcroft2008, linz2006}, pode-se definir uma série de operações sobre palavras, sendo a primeira delas  a noção de concatenação.

\begin{definition}[Concatenação de palavras]\label{def:Concatenacao}
	Sejam $w_1 = a_1\cdots a_m$ e $w_2 = b_1\cdots b_n$ duas palavras quaisquer, tem-se que a concatenação de $w_1$ e $w_2$, denotado por $w_1w_2$, corresponde a uma sequência iniciada com os símbolos que forma $w_1$ imediatamente seguido dos símbolos que forma $w_2$, ou seja, $w_1w_2 = a_1\cdots a_mb_1\cdots b_n$.
\end{definition}

\begin{rema}
	O leitor deve ficar atento ao fato de que a concatenação apenas combina duas palavras em uma nova palavra, sendo que, não a qualquer tipo de exigência sobre os alfabeto sobre os quais as palavras usadas na concatenação estão definidas.
\end{rema}

\begin{exem}\label{exe:Concatenacao}
	Dado duas palavras $w_1 = abra$ e $w_2 = cadabra$ tem-se que $w_1w_2 = abracadabra$ e $w_2w_1 = cadabraabra$.
\end{exem}

Note que o Exemplo \ref{exe:Concatenacao} estabelece que a operação de concatenação entre duas palavras não é comutativa, isto é, a ordem com que as palavras aparecem na concatenação é responsável pela forma da palavra resultante da concatenação.

\begin{theorem}[Associativade da Concatenação]\label{teo:AssociatividaeConcatenacao}
	Para quaisquer $w_1, w_2$ e $w_3$ tem-se que $(w_1w_2)w_3 = w_1(w_2w_3)$.
\end{theorem}

\begin{proof}
	Dado três palavras quaisquer $w_1 = a_1\cdots a_i, w_2 = b_1\cdots b_j$ e $w_3 = c_1\cdots c_k$ tem-se que,
	\begin{eqnarray*}
		(w_1w_2)w_3 & = & (a_1\cdots a_ib_1\cdots b_j)c_1\cdots c_k\\
		& = & a_1\cdots a_ib_1\cdots b_jc_1\cdots c_k\\
		& = & a_1\cdots a_i(b_1\cdots b_jc_1\cdots c_k)\\
		& = & w_1(w_2w_3)
	\end{eqnarray*}
	o que conclui a prova.
\end{proof}

Sobre qualquer alfabeto $\Sigma$ sempre é definida uma palavra especial chamada \textbf{palavra vazia} \cite{hopcroft2008, linz2006}, esta palavra especial não possui nenhum símbolo, e em geral é usado o símbolo $\lambda$ para denotar a palavra vazia \cite{benjaLivro2010, valdi2016master}. Como destacado em \cite{benjaLivro2010, valdi2020phd} sobre a palavra vazia é importante destacar que:

\begin{eqnarray}
	w\lambda & = & \lambda w = w\\
	|\lambda| & = &  0
\end{eqnarray}

Isto é, a palavra vazia é neutra para a operação de concatenação, além disso, a mesma apresenta comprimento nulo.

\begin{definition}[Potência das palavras]\label{def:PotenciaPalavras}
	Seja $w$ uma palavra sobre um alfabeto $\Sigma$ a potência de $w$ é definida recursivamente para todo $n \in \mathbb{N}$ como sendo:
	\begin{eqnarray}
		w^0 & = & \lambda\\
		w^{n+1} & = & ww^{n}
	\end{eqnarray}
\end{definition}

\begin{exem}
	Sejam $w_1 = ab, w_2 = bac$ e $w_3 = cbb$ palavras sobre $\Sigma = \{a, b, c\}$ tem-se que:
	\begin{itemize}
		\item[(a)] $w_1^3 = w_1w_1^2 = w_1w_1w_1^1 = w_1w_1w_1w_1^0 = w_1w_1w_1\lambda = ababab$.
		\item[(b)] $w_2^2 = w_2w_2^1 = w_2w_2w_2^0 = w_2w_2\lambda = w_2w_2 = bacbac$.
	\end{itemize} 
\end{exem}

\begin{exem}
	Seja $u = 01$ e $v = 231$ tem-se que: 
	$$uv^3 = uvv^2 = uvvv^1 = uvvv\lambda = uvvv = 01231231231$$
	e também 
	$$u^2v = uu^1v = uu\lambda v = uuv = 0101231$$
\end{exem}

\begin{prop}
	Para toda palavra $w$ e todo $m,n \in \mathbb{N}$ tem-se que:
	\begin{itemize}
		\item[(i)] $(w^m)^n = w^{mn}$.
		\item[(ii)] $w^mw^n = w^{m+n}$.
	\end{itemize}
\end{prop}

\begin{proof}
	Direto das Definições \ref{def:Concatenacao} e \ref{def:PotenciaPalavras}, e portanto, ficará como exercício ao leitor.
\end{proof}

Outro importante conceito existente sobre a ideia de palavra é a noção de palavra inversa formalmente definida como se segue.

\begin{definition}[Palavra Inversa]
	\cite{valdi2016master} Seja $w = a_1\cdots a_n$ uma palavra qualquer, a palavra inversa de $w$ denotada por $w^r$, é tal que $w^r = a_n\cdots a_1$. 
\end{definition}

\begin{exem}
	Dado as palavras $u = aba, v = 011101$ e $w = 3021$ tem-se que $u^r = aba, v^r = 101110$ e $w^r = 1203$.
\end{exem}

\begin{rema}
	Com respeito a noção de palavra inversa tem-se em particular que vale a seguinte igualdade $\lambda^r = \lambda$.
\end{rema}

Além das palavras, pode-se também formalizar uma série de operações sobre a própria noção de alfabeto. Em primeiro lugar, uma vez que,  alfabetos são conjuntos, obviamente todas operações usuais de união, interseção, complemento, diferença e diferença simétrica (ver Capítulo \ref{cap:Conjuntos}) também são válidas sobre alfabetos. Além dessas operações, também esta definida a operação de potência e os fechos positivo e de Kleene sobre alfabetos.

\begin{definition}[Potência]
	\cite{benjaLivro2010} Seja $\Sigma$ um alfabeto a potência de $\Sigma$ é definida recursivamente para todo $n \in \mathbb{N}$ como:
	\begin{eqnarray}
		\Sigma^0 & = & \{\lambda\}\\
		\Sigma^{n+1} & = & \{aw \mid a \in \Sigma, w \in \Sigma^{n}\}
	\end{eqnarray}
\end{definition} 

\begin{exem}
	Dado $\Sigma = \{a, b\}$ tem-se que $\Sigma^3 = \{aaa, aab, aba, baa, abb, bab, bba, bbb\}$ e $\Sigma^1 = \{a, b\}$
\end{exem}

\begin{exem}
	Seja $\Sigma = \{0, 1, 2\}$ tem-se que $\Sigma^2 = \{00, 01, 02, 10, 11, 12, 20, 21, 22\}$ e $\Sigma^{0} = \{\lambda\}$.
\end{exem}

O leitor mais atencioso e maduro matematicamente pode notar que para qualquer que seja $n \in \mathbb{N}$ o conjunto potência tem a propriedade de que todo $w \in \Sigma^n$ é tal que $|w| = n$, além disso, é claro que todo $\Sigma^n$ é sempre finito\footnote{Essa afirmação é facilmente verificável, uma vez que, a mesma nada mais é do que um exemplo de arranjo com repetição.}.

\begin{definition}[Fechos]\label{def:FechoPositivoKleene}
	Seja $\Sigma$ um alfabeto o fecho positivo e o fecho de Kleene de $\Sigma$, denotados respectivamente por $\Sigma^+$ e $\Sigma^*$, correspondem aos conjuntos:
	\begin{eqnarray}
		\Sigma^+ & = & \bigcup_{i = 1}^\infty \Sigma^i
	\end{eqnarray}
	e
	\begin{eqnarray}
		\Sigma^* & = & \bigcup_{i = 0}^\infty \Sigma^i
	\end{eqnarray}
\end{definition}

Obviamente como dito em \cite{benjaLivro2010}, o fecho de positivo pode ser reescrito em função do fecho de Kleene usando a operação de diferença de conjunto, isto é, o fecho positivo corresponde a seguinte identidade, $\Sigma^+ = \Sigma^* - \{\lambda\}$. Sobre o fecho de Kleene com destacado em \cite{valdi2020phd} o mesmo corresponde ao monoide livremente\footnote{Relembre que uma álgebra é livremente gerada quando todo elemento possui fatoração única (a menos de isomorfismo).} gerado pelo conjunto $\Sigma$ munida da operação de concatenação.

\begin{definition}[Prefixos e Sufixos]
	Uma palavra $u \in \Sigma^*$ é um prefixo de outra palavra $w \in \Sigma^*$, denotado por $u \preceq_p w$, sempre que $w = uv$, com $v \in \Sigma^*$. Por outro lado, uma palavra $u$ é um sufixo de outra palavra $w$, denotado por $u \preceq_s w$, sempre que $w = vu$.
\end{definition}

\begin{exem}
	Seja $w = abracadabra$ tem-se qu~e as palavras $ab$ e $abrac$ são prefixos de $w$, por outro, lado $cadabra$ e $bra$ são sufixos de $w$, e a palavra $abra$ é prefixo e também sufixo. Já a palavra $cada$ não é prefixo e nem sufixo de $w$.
\end{exem}


\begin{definition}
	Seja $w \in \Sigma^*$ o conjunto de todos os prefixos de $w$ corresponde ao conjunto:
	\begin{eqnarray}
		PRE(w) = \{w' \in \Sigma^* \mid w' \preceq_p w\}
	\end{eqnarray}
	e o conjunto de todos os sufixos de $w$ corresponde ao conjunto:
	\begin{eqnarray}
		SUF(w) = \{w' \in \Sigma^* \mid w' \preceq_s w\}
	\end{eqnarray}
\end{definition}

\begin{exem}
	Seja $w = univasf$ tem-se que:
	\begin{eqnarray*}
		PRE(w) = \{\lambda, u, un, uni, univ, univa, univas, univasf\}
	\end{eqnarray*}
	e
	\begin{eqnarray*}
		SUF(w) = \{\lambda, f, sf, asf, vasf, ivasf, nivasf, univasf \}
	\end{eqnarray*}
\end{exem}

\begin{exem}
	A seguir é apresentado alguns exemplos de palavras e seus conjuntos de prefixos e sufixos.
	\begin{itemize}
		\item[(a)] Se $w = ab$, então $PRE(w) = \{\lambda, a, ab\}$ e  $SUF(w) = \{\lambda, b, ab\}$.
		\item[(b)] Se $w = 001$, então $PRE(w) = \{\lambda, 0, 00, 001\}$ e  $SUF(w) = \{\lambda, 1, 01, 001\}$.
		\item[(c)] Se $w = \lambda$, então $PRE(w) = \{\lambda\}$ e  $SUF(w) = \{\lambda\}$
		\item[(d)] Se $w = a$, então $PRE(w) = \{\lambda, a\}$ e $SUF(w) = \{\lambda, a\}$.
	\end{itemize}
\end{exem}

Com respeito a cardinalidade dos conjuntos de prefixos e sufixos, os mesmo apresentam as propriedades descritas pelo teorema a seguir.

\begin{theorem}\label{teo:CardinalidadePrefixoSufixo}
	Para qualquer que seja $w \in \Sigma^*$ as seguintes asserções são verdadeiras.
	\begin{itemize}
		\item[(i)] $\# PRE(w) = |w| + 1$.
		\item[(ii)] $\#PRE(w) = \#SUF(w)$.
		\item[(iii)] $\#(PRE(w) \cap SUF(w)) \geq 1$.
	\end{itemize}
\end{theorem}

\begin{proof}
	Dado uma palavra $w$ tem-se que:
	\item[(i)] Sem perda de generalidade assumindo que $w = a_1\cdots a_n$ logo $w \in \Sigma^n$ (o caso quando $w = \lambda$ é trivial e não será demonstrado aqui) logo $|w| = n$ para algum $n \in \mathbb{N}$, assim existem exatamente $n$ palavras da forma $a_1 \cdots a_i$ com $1 \leq i \leq n$ tal que $a_1 \cdots a_i \preceq_p w$, portanto, para todo $1 \leq i \leq n$ tem-se que $a_1 \cdots a_i \in PRE(w)$, além disso, é claro que $w = \lambda w$, e portanto, $\lambda \in PRE(w)$, consequentemente, $\#PRE(w) = n + 1 = |w| + 1$.
	\item[(ii)] É suficiente mostrar que $\# SUF(w) = |w| + 1$, para isso como antes sem perda de generalidade assuma que $w = a_1\cdots a_n$ e assim tem-se que $w \in \Sigma^n$ logo $|w| = n$ com $n \in \mathbb{N}$, dessa forma existem exatamente $n$ palavras da forma $a_i \cdots a_n$ com $1 \leq i \leq n$ tal que $a_i \cdots a_n \preceq_s w$, portanto, para todo $1 \leq i \leq n$ tem-se que $a_i \cdots a_n \in SUF(w)$, além disso, é claro que $w = w\lambda$, logo $\lambda \in SUF(w)$, consequentemente, $\#SUF(w) = n + 1 = |w| + 1$, e portanto, $\#PRE(w) = \#SUF(w)$. O caso $w = \lambda$ é trivial e não será demonstrado aqui.
	\item[(iii)] Trivial, pois basta notar que $\lambda \in (PRE(w) \cap SUF(w))$, e portanto, tem-se claramente que $\#(PRE(w) \cap SUF(w)) \geq 1$.
\end{proof}

\begin{corollary}
	Toda palavra tem pelo menos um prefixo e um sufixo.
\end{corollary}

\begin{proof}
	Direto do item $(iii)$ do Teorema \ref{teo:CardinalidadePrefixoSufixo}.
\end{proof}


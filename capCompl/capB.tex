% Seta a imagem do capítulo
\chapterimage{chapter_head_2.pdf}
% O título e rótulo do cabiluto
\chapter{Revisão Matemática}\label{cap:IntroducaoComplexidade}


\epigraph{``A ciência é o que nós compreendemos suficientemente bem para explicar a um computador. A arte é tudo mais.''}{Donald Knuth.}

\section{Somatórios: Notação e Definições Básicas}

Como dito em \cite{carmo2013}, os somatórios apresentam um papel de suma importância em diversos campos da matemática e demais ciência exatas. Para o estudo da análise de algoritmos que será feitos em capítulos futuros o domínio de tais objetos matemáticos (os somatórios) será de extrema necessidade, assim este capítulo de revisão irá se iniciar reforçando ao leitor as notações e propriedades dos somatórios.

O leitor assim como todo estudante deve se lembrar de suas aulas na escola primária em que a operação de soma lhe foi apresenta como sendo uma operação\footnote{Nos termos mais formais abordados neste manuscrito até agora, pode-se repensar a noção de soma como sendo uma função cuja assinatura é a da forma $+: A \times A \rightarrow A$ onde $A$ é um conjunto numérico qualquer.} em que dois números eram dados como entrada e um número deveria ser o resultado da ``combinação'' dos números da entrada.